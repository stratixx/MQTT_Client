\documentclass[a4paper,titlepage,11pt,twosides,floatssmall]{mwrep}
\usepackage[left=2.5cm,right=2.5cm,top=2.5cm,bottom=2.5cm]{geometry}
\usepackage[OT1]{fontenc}
\usepackage{polski}
\usepackage{amsmath}
\usepackage{amsfonts}
\usepackage{amssymb}
\usepackage{graphicx}
\usepackage{url}
\usepackage{tikz}
\usepackage{float}
\usetikzlibrary{arrows,calc,decorations.markings,math,arrows.meta}
\usepackage{rotating}
\usepackage[percent]{overpic}
\usepackage[utf8]{inputenc}
\usepackage{xcolor}
\usepackage{pgfplots}
\usetikzlibrary{pgfplots.groupplots}
%klikalne odnoścniki w spisie treści
\usepackage{color}   %May be necessary if you want to color links
\usepackage[pdfencoding=auto]{hyperref}
\hypersetup{
    colorlinks=true, %set true if you want colored links
    linktoc=all,     %set to all if you want both sections and subsections linked
	linkcolor=black,  %choose some color if you want links to stand out
	bookmarksopen=true, % 
}

%wiele kolumn tekstu na stronie
\usepackage{multicol}

\usepackage{listings}
\usepackage{matlab-prettifier}
\usepackage{enumitem,amssymb}
\definecolor{szary}{rgb}{0.95,0.95,0.95}
\usepackage{siunitx}
\sisetup{detect-weight,exponent-product=\cdot,output-decimal-marker={,},per-mode=symbol,binary-units=true,range-phrase={-},range-units=single}
\SendSettingsToPgf
%konfiguracje pakietu listings
\lstset{
	backgroundcolor=\color{szary},
	frame=single,
	breaklines=true,
}
\lstdefinestyle{customlatex}{
	basicstyle=\footnotesize\ttfamily,
	%basicstyle=\small\ttfamily,
}
\lstdefinestyle{customc}{
	breaklines=true,
	frame=tb,
	language=C,
	xleftmargin=0pt,
	showstringspaces=false,
	basicstyle=\small\ttfamily,
	keywordstyle=\bfseries\color{green!40!black},
	commentstyle=\itshape\color{purple!40!black},
	identifierstyle=\color{blue},
	stringstyle=\color{orange},
}
\lstdefinestyle{custommatlab}{
	captionpos=t,
	breaklines=true,
	frame=tb,
	xleftmargin=0pt,
	language=matlab,
	showstringspaces=false,
	%basicstyle=\footnotesize\ttfamily,
	basicstyle=\scriptsize\ttfamily,
	keywordstyle=\bfseries\color{green!40!black},
	commentstyle=\itshape\color{purple!40!black},
	identifierstyle=\color{blue},
	stringstyle=\color{orange},
}

%wymiar tekstu (bez ?ywej paginy)
\textwidth 160mm \textheight 247mm

% ustawienia pakietu pgfplots
\pgfplotsset{
	compat=1.16, % tryb kompatybilności
	tick label style={font=\scriptsize},
	label style={font=\small},
	legend style={font=\small},
	title style={font=\small}
}

% ustawienia znaczników kolejnych poziomów list typu itemize
\renewcommand\labelitemi{\textbullet} % duża kropka
\renewcommand\labelitemii{\textasteriskcentered} % gwiazdka
\renewcommand\labelitemiii{\textperiodcentered} % 
\renewcommand\labelitemiv{\textendash} % 


\def\figurename{Rys.}
\def\tablename{Tab.}

%konfiguracja liczby p?ywaj?cych element?w
\setcounter{topnumber}{0}%2
\setcounter{bottomnumber}{3}%1
\setcounter{totalnumber}{5}%3
\renewcommand{\textfraction}{0.01}%0.2
\renewcommand{\topfraction}{0.95}%0.7
\renewcommand{\bottomfraction}{0.95}%0.3
\renewcommand{\floatpagefraction}{0.35}%0.5

\begin{document}
\frenchspacing
\pagestyle{uheadings}

%strona tytu?owa
\title{\bf Sprawozdanie końcowe \vskip 0.1cm}
\author{Marcin Dolicher \\ \indent Konrad Winnicki}
\date{7 czerwca 2019}

\makeatletter
\renewcommand{\maketitle}{\begin{titlepage}
\begin{center}
{\LARGE {\bf Politechnika Warszawska}}\\
\vspace{0.4cm}
{\LARGE {\bf Wydział Elektroniki i Technik Informacyjnych}}\\
\vspace{0.2cm}
{\LARGE {\bf Instytut Automatyki i Informatyki Stosowanej}}\\
\end{center}
\vspace{5cm}
\begin{center}
{\bf \LARGE Zaawansowane programowanie w C++ \vskip 0.1cm}
\end{center}
\vspace{1cm}
\begin{center}
{\bf \LARGE \@title}
\end{center}
\vspace{9cm}
%\begin{center}
%{\bf Autor: \hspace{2cm} Promotor:  \\ \@author \hspace{2cm} Winiar}
%{\bf Autor:\\ \Large \@author \par \hspace{2cm} Promotor: \@promotor}
%\end{center}
\begin{multicols}{2}
	\bf \large Autorzy:\\ \Large \indent	\@author \par
	\vfill\null
	\columnbreak
	\bf \large Prowadzący projekt:\\ \Large \indent	mgr inż. Konrad Grochowski \par
\end{multicols}
\vspace*{\stretch{6}}
\begin{center}
\bf{\large{Warszawa, \@date\vskip 0.1cm}}
\end{center}
\end{titlepage}
}
\makeatother

\maketitle

\tableofcontents
%%%%%%%%%%%%%%%%%%%%%%%%%%%%%%%%%%%%%%%%%%%%%%%%%%%%%%
\chapter{Dokumentacja użytkownika}

Dokumentację kodu można wygenerować przy pomocy narzędzia doxygen, które na podstawie komentarzy
w kodzie tworzy dokumentację.



%%%%%%%%%%%%%%%%%%%%%%%%%%%%%%%%%%%%%%%%%%%%%%%%%%%%%%%
\chapter{Statystyki}
%%%%%%%%%%%%%%%
\section{Liczba linii kodu}

%\subsection{}

\begin{itemize}
	\item Biblioteka MQTT Client
\begin{itemize}

\item $MQTT\_Client.cpp$ - 229 linijek 
\item $Connection.h$ - 36 linijek 
\item $ConnectionUnscripted.h$ - 49 linijek 
\item $MQTT\_Client.hpp$ - 147 linijek 
\item $SocketHandle.cpp$ - 48 linijek
\item $Connection.cpp$ - 25 linijek 
\item $ConnectionUnscripted.cpp$ - 105 linijek 
\item Łącznie 639 linijek
\end{itemize}

%\subsection{Biblioteka DataStore}
	\item Biblioteka DataStore
\begin{itemize}
\item $Data\_Archive.cpp$ - 25 linijek
\item $DataJSON.cpp$ - 35 linijek 
\item $DataStore.cpp$ - 25 linijek 
\item $DataArchive.hpp$ - 30 linijek 
\item $DataJSON.hpp$ - 25 linijek 
\item $DataStore.hpp$ - 34 linijek 
\item Łącznie 174 linijek
\end{itemize}

	\item Program Subscriber
%\subsection{Program Subscriber}

\begin{itemize}
\item $Subscriber.cpp$ - 54 linijki
\item $SubscriberData.cpp$ - 15 linijek
\item $subscriber\_test\_1.cpp$ - 14 linijek 
\item $subscriber\_test\_2.cpp$ - 14 linijek 
\item $Subscriber.hpp$ - 15 linijek 
\item $SubcriberData.hpp$ - 8 linijek 
\item Łącznie 120 linijek
\end{itemize}

	\item Program Publisher
%\subsection{Program Publisher}

\begin{itemize}
\item $Publisher.cpp$ - 41 linijek 
\item $publiser\_test\_1.cpp$ - 14 linijek 
\item $publiser\_test\_2.cpp$ - 14 linijek 
\item Łącznie 69 linijek
\end{itemize}

\end{itemize}
Łącznie 1000 linijek kodu

%%%%%%%%%%%%%
\section{Liczba testów}

\begin{itemize}
	\item Biblioteka MQTT Client
%\subsection{Biblioteka MQTT Client}

Przygotowano dwa testy jednostkowe testujące próbę połączenia. 
\newline
Pierwszy test sprawdza próbę połaczenia wspieranego - nieszyfrowanego.
\newline
Drugi test sprawdza próbę połączenia niespieranego - szyfrowanego.

	\item Biblioteka DataStore
%\subsection{Biblioteka DataStore}

Przygotowano dwa testy jednostkowe pokazujące pomyślne skonfigurowanie środowiska.

	\item Program Subscriber
\subsection{Program Subscriber}

Przygotowano dwa testy jednostkowe pokazujące pomyślne skonfigurowanie środowiska.

\subsection{Program Publisher}

Przygotowano dwa testy jednostkowe pokazujące pomyślne skonfigurowanie środowiska.


%%%%%%%%%%%%%
\section{Procentowe pokrycie kodu testami}

\subsection{Biblioteka MQTT Client}

Testy pokrywają jedną z ośmiu funkcji biblioteki, daje to \num{12.5} procent pokrycia.

\subsection{Biblioteka DataStore}

Testy pokrywają jedną z ośmiu funkcji biblioteki, daje to \num{12.5} procent pokrycia.

\subsection{Program Subscriber}

Testy pokrywają jedną z ośmiu funkcji biblioteki, daje to \num{12.5} procent pokrycia.

\subsection{Program Publisher}

Testy pokrywają jedną z ośmiu funkcji biblioteki, daje to \num{12.5} procent pokrycia.

%%%%%%%%%%%
\section{Liczba godzin poświęcona na projekt}

Zespół powięcił na projekt łącznie około 120 godzin.
\newline
Większą część poświęcono na konfigurację środowiska.

%%%%%%%%%%%%%%%%%%%%%%%%%%%%%%%%%%%%%%%%%%%%%%%%%%%%%%%%%%%%%%%5
\chapter{Opis napotkanych problemów i popełnionych błędów}

Podczas projektu napotkaliśmy spore problemu z dołączaniem bibliotek za pomocą cmake do projektu. 
Spowodowało to spore utrudnienia w pracy, które wygenerowały duże opóźnienia. 
Początkowym problemem było zrozumienie i zaprojektowanie prawidłowej architektury dla sniffera MQTT. 
Nie przewidzieliśmy problemów z samą implementacją protokołu MQTT. 
Dotyczyły one zrozumienia dokumentacji i praktycznego zastosowania informacji w niej zawartych. 
%%%%%%%%%%%%%%%%%%%%%%%%%%%%%%%%%%%%%%%%%%%%%%%%%%%%%%%%%%%%%


\end{document}

